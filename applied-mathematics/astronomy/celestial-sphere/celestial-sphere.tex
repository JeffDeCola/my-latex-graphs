\documentclass[border=5mm]{standalone}
\usepackage{tikz}
\usepackage{tikz-3dplot}
\usetikzlibrary{positioning}

\begin{document}

% DEFINE YOUR 3D COORDINATE FRAME / ORIENTATION (using tikz-3dplot)
% Will set the orientation of x,y,z
% Set polar coordinate. theta ( z -> phi) phi (x -> y) 
% Syntax: \tdplotsetdisplay{\theta_d}{\phi_d}
\tdplotsetmaincoords{70}{150}

% SET SOME VARIABLES

    % Sphere radius
    \pgfmathsetmacro{\sphereRadius}{1}

    % Set Polar (2D) Coordinates for Ecliptic
    \pgfmathsetmacro{\vectorRadiusEcliptic}{0}
    \pgfmathsetmacro{\vectorThetaEcliptic}{23.5}
    \pgfmathsetmacro{\vectorPhiEcliptic}{0}

    % Set Polar (2D) Coordinates for Star
    \pgfmathsetmacro{\vectorRadiusStar}{1}
    \pgfmathsetmacro{\vectorThetaSar}{25}
    \pgfmathsetmacro{\vectorPhiStar}{45}

% USE tdplot_main_coords FOR YOUR DEFINED 3D COORDINATE FRAME
% Coordinate transformation provided by tikz-3dplot
\begin{tikzpicture}[scale=6, tdplot_main_coords]

    % DEFINE coordinates - the origin
    \coordinate (Origin) at (0,0,0);
    \coordinate (northCelestialPole) at (0,0,\sphereRadius);
    \coordinate (southCelestialPole) at (0,0,-\sphereRadius);
    \coordinate (springEquinox) at (0,\sphereRadius,0);
    \coordinate (autumnEquinox) at (0,-\sphereRadius,0);

    % DRAW CELESTIAL SPHERE ON 3D COORDINATE SYSTEM
    \shadedraw[tdplot_screen_coords] (0,0) circle (\sphereRadius);

    % DRAW MAIN COORDINATE SYSTEM AXES - X, Y, Z AXIS WITH LABEL (NODE)
    % Syntax: \draw[characteristics] (start point) -- (end point) node [anchor=position] {label};
    \draw[thick, color=black, ->] (Origin) -- (\sphereRadius+.5,0,0) node[anchor=north east] {$x$};
    \draw[thick, color=black, ->] (Origin) -- (0,\sphereRadius+.5,0) node[anchor=north west] {$y$};
    \draw[thick, color=black, ->] (Origin) -- (0,0,\sphereRadius+.5) node[anchor=south] {$z$};
    \draw[dashed, color=black, ->] (Origin) -- (-\sphereRadius-.5,0,0);
    \draw[dashed, color=black, ->] (Origin) -- (0,-\sphereRadius-.5,0);
    \draw[dashed, color=black, ->] (Origin) -- (0,0,-\sphereRadius-.5);

    % DRAW GREEN CELESTIAL EQUATOR & LABEL
    % \draw[thin, fill, color=white] (\sphereRadius,0,0) arc (0:360:\sphereRadius);
    \draw[dashed, thin, color=green!60!black] (\sphereRadius,0,0) arc (0:360:\sphereRadius);
    \draw[thin, color=green!60!black] (\sphereRadius,0,0) arc (0:-20:\sphereRadius);
    \draw[thin, color=green!60!black] (\sphereRadius,0,0) arc (0:150:\sphereRadius);
    \node at (1,.7,0) [color=green!60!black] {Celestial Equator};
    \node at (\sphereRadius,0,0) [circle, color=green!60!black, fill, inner sep=1.5pt] {};
    \node at (-\sphereRadius,0,0) [circle, color=green!60!black, fill, inner sep=1.5pt] {};

    % LABEL CELESTIAL POLES (WITH BLACK DOTS)
    \node at (northCelestialPole) [circle, color=black, fill, inner sep=1.5pt] {};
    \node at (northCelestialPole) [color=black, above right = 6mm] {North Celestial Pole};
    \node at (southCelestialPole) [circle, color=black, fill, inner sep=1.5pt] {};
    \node at (southCelestialPole) [color=black, below right = 6mm] {South Celestial Pole};

    % DRAW EARTH & LABEL EARTH
    \draw[thick] (0,0) circle (0.1cm);
    \shade[ball color=blue!60!white, opacity=0.80] (0,0) circle (0.1cm);
    \node at (Origin) [color=black, above left = 5mm of Origin] {Earth};

    % SET ECLIPTIC COORDINATES
    % Set the rotated coordinate definition within display using a translation
    % Coordinate and Euler angles in the "z(\alpha)y(\beta)z(\gamma)" euler rotation convention
    % Syntax: \tdplotsetrotatedcoords{\alpha}{\beta}{\gamma}
    \tdplotsetrotatedcoords{\vectorPhiEcliptic}{\vectorThetaEcliptic}{0}

    % DRAW RED ECLIPTIC & LABEL
    % \draw[thin, fill, color=white] (\sphereRadius,0,0) arc (0:360:\sphereRadius);
    \draw[tdplot_rotated_coords, dashed, thin, color=red!80!black] (\sphereRadius,0,0) arc (0:360:\sphereRadius);
    \draw[tdplot_rotated_coords, thin, color=red!80!black] (\sphereRadius,0,0) arc (0:-20:\sphereRadius);
    \draw[tdplot_rotated_coords, thin, color=red!80!black] (\sphereRadius,0,0) arc (0:160:\sphereRadius);
    \node at (.9,.7,0) [tdplot_rotated_coords, color=red!80!black] {Ecliptic};

    % LABEL ECLIPTIC (EQUINOX) POLES (WITH RED DOTS)
    \node at (springEquinox) [tdplot_rotated_coords, circle, color=red!80!black, fill, inner sep=1.5pt] {};
    \node at (springEquinox) [tdplot_rotated_coords, color=red!80!black, below = 6mm] {Spring Equinox};
    \node at (autumnEquinox) [tdplot_rotated_coords, circle, color=red!80!black, fill, inner sep=1.5pt] {};
    \node at (autumnEquinox) [tdplot_rotated_coords, color=red!80!black, above = 6mm] {Autumn Equinox};
    
\end{tikzpicture}

\end{document}