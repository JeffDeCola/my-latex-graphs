\documentclass[border=5mm]{standalone}
\usepackage{tikz}
\usepackage{tikz-3dplot}
\usetikzlibrary{positioning}

\begin{document}

% DEFINE YOUR 3D COORDINATE FRAME / ORIENTATION (using tikz-3dplot)
    
    % Will set the orientation of x,y,z
    % Set polar coordinate.
    % At (0,0) z pointing at user, x,y are normal.
    % - phi (x -> y) Rotated around z
    % - theta (z -> phi)
    % Syntax: \tdplotsetdisplay{\theta_d}{\phi_d}
    \pgfmathsetmacro{\phi}{50}
    \pgfmathsetmacro{\theta}{70}
    \tdplotsetmaincoords{\theta}{\phi}

% SET SOME VARIABLES

    % CELESTIAL SPHERE RADIUS
    \pgfmathsetmacro{\sphereRadius}{1}

    % EARTH RADIUS
    \pgfmathsetmacro{\earthRadius}{.15}

    % ECLIPTIC 2D POLAR ANGLES FOR 3D COORDINATES
    % - degrees rotated on the x-axis (Counter Clockwise)
    % - 23.5 degrees rotated up from on xy-plane
    \pgfmathsetmacro{\vectorRadiusEcliptic}{0}
    \pgfmathsetmacro{\vectorPhiEcliptic}{90}
    \pgfmathsetmacro{\vectorThetaEcliptic}{-23.5}

    % STAR (RIGHT ASCENSION & DECLINATION)
    \pgfmathsetmacro{\starRightAscension}{20}
    \pgfmathsetmacro{\starDeclination}{60}
    % At (0,0) Star sitting on x (Spring Equinox)
    % - Right Ascension degrees rotated on the x-axis (Counter Clockwise)
    % - Declination degrees rotated up from on xy-plane
    % 2D POLAR ANGLES FOR 3D COORDINATES
    \pgfmathsetmacro{\vectorRadiusStar}{1}
    \pgfmathsetmacro{\vectorPhiStar}{\starRightAscension}
    \pgfmathsetmacro{\vectorThetaStar}{90-\starDeclination}

    % OBSERVER & ZENITH (MAP IT LIKE (RIGHT ASCENSION & DECLINATION)
    \pgfmathsetmacro{\zenithRightAscension}{280}
    \pgfmathsetmacro{\zenithDeclination}{70}
    % At (0,0) Zenith sitting on x (Spring Equinox)
    % - Right Ascension degrees rotated on the x-axis (Counter Clockwise)
    % - Declination degrees rotated up from on xy-plane
    % 2D POLAR ANGLES FOR 3D COORDINATES
    \pgfmathsetmacro{\vectorRadiusZenith}{1}
    \pgfmathsetmacro{\vectorPhiZenith}{\zenithRightAscension}
    \pgfmathsetmacro{\vectorThetaZenith}{90-\zenithDeclination}

    % MERIDIAN
    % Perpendicular to Celestial Equator and pass through Zenith
    % - zenithRightAscension degrees rotated on z-axis (Counter Clockwise)
    % - 90 degrees rotated on y-axis (Counter Clockwise)
    % 2D POLAR ANGLES FOR 3D COORDINATES
    \pgfmathsetmacro{\vectorRadiusMeridian}{1}
    \pgfmathsetmacro{\vectorPhiMeridian}{\zenithRightAscension}
    \pgfmathsetmacro{\vectorThetaMeridian}{90}

% USE tdplot_main_coords FOR YOUR DEFINED 3D COORDINATE FRAME
% Coordinate transformation provided by tikz-3dplot
\begin{tikzpicture}[scale=8, tdplot_main_coords]

    % DEFINE MAIN COORDINATES

        \coordinate (Origin) at (0,0,0);
        \coordinate (northCelestialPole) at (0,0,\sphereRadius);
        \coordinate (southCelestialPole) at (0,0,-\sphereRadius);
        \coordinate (springEquinox) at (\sphereRadius,0,0);
        \coordinate (autumnEquinox) at (-\sphereRadius,0,0);

    % CELESTIAL SPHERE & A 3D AXIS (SHADING)

        % DRAW CELESTIAL SPHERE ON MAIN 3D COORDINATE SYSTEM
        \shadedraw[tdplot_screen_coords, thin, color=black, outer color=black!70, inner color=white] (0,0) circle (\sphereRadius);

        % DRAW MAIN COORDINATE SYSTEM AXES. X, Y, Z AXIS WITH LABELS (BLACK)
        % Syntax: \draw[characteristics] (start point) -- (end point) node [anchor=position] {label};
        \draw[thick, color=black, ->] (Origin) -- (\sphereRadius+.5,0,0) node[anchor=west] {$x$};
        \draw[thick, color=black, ->] (Origin) -- (0,\sphereRadius+.5,0) node[anchor=west] {$y$};
        \draw[thick, color=black, ->] (Origin) -- (0,0,\sphereRadius+.5) node[anchor=south] {$z$};
        \draw[dashed, color=black, ->] (Origin) -- (-\sphereRadius-.5,0,0);
        \draw[dashed, color=black, ->] (Origin) -- (0,-\sphereRadius-.5,0);
        \draw[dashed, color=black, ->] (Origin) -- (0,0,-\sphereRadius-.5);

        % DRAW AND LABEL CELESTIAL POLES (GREEN DOTS)
        % North and South Poles
        \node at (northCelestialPole) [circle, color=green!50!black, fill, inner sep=1.5pt] {};
        \node at (northCelestialPole) [color=black, above right = 14mm] {North Celestial Pole};
        \node at (southCelestialPole) [circle, color=green!50!black, fill, inner sep=1.5pt] {};
        \node at (southCelestialPole) [color=black, below right = 14mm] {South Celestial Pole};

    % EARTH (BLUE)

        % DRAW EARTH & LABEL
        \draw[thick] (Origin) circle (\earthRadius cm);
        \shade[ball color=blue!60!white, opacity=0.40] (Origin) circle (\earthRadius cm);
        \node at (Origin) [color=black, below left = 5mm of Origin] {Earth};

    % CELESTIAL EQUATOR (GREEN)

        % DRAW CELESTIAL EQUATOR & LABEL (GREEN)
        \draw[dashed, thick, color=green!80!black] (\sphereRadius,0,0) arc (0:360:\sphereRadius);
        % Note that I use \phi to figure out the think line :)
        \tdplotdrawarc[very thick, color=green!80!black]{(Origin)}{\sphereRadius}{-180+\phi}{\phi}{};
        \node at (-.9,-.9,0) [color=green!80!black] {Celestial Equator};

        % (GREEN DOTS)
        \node at (0, \sphereRadius,0) [circle, color=green!80!black, fill, inner sep=1.5pt] {};
        \node at (0, -\sphereRadius,0) [circle, color=green!80!black, fill, inner sep=1.5pt] {};

    % ECLIPTIC & EQUINOX (RED)

        % SET ECLIPTIC COORDINATES
        % Set the rotated coordinate definition within display using a translation
        % Coordinate and Euler angles in the "z(\alpha)y(\beta)z(\gamma)" euler rotation convention
        % Syntax: \tdplotsetrotatedcoords{\alpha}{\beta}{\gamma}
        \tdplotsetrotatedcoords{\vectorPhiEcliptic}{\vectorThetaEcliptic}{0}

        % DRAW ECLIPTIC & LABEL (RED)
        \draw[tdplot_rotated_coords, dashed, thick, color=red!80!black] (\sphereRadius,0,0) arc (0:360:\sphereRadius);
        % Note that I use \phi to figure out the think line :)
        \tdplotdrawarc[tdplot_rotated_coords, very thick, color=red!80!black]{(Origin)}{\sphereRadius}{-180+\phi-90}{\phi-90}{};
        \node at (-1.3,0,.3) [tdplot_rotated_coords, color=red!80!black] {Ecliptic};

        % SPRING AND AUTUMN EQUINOX & LABEL (RED DOTS)
        \node at (springEquinox) [tdplot_rotated_coords, circle, color=red!80!black, fill, inner sep=1.5pt] {};
        \node at (springEquinox) [tdplot_rotated_coords, color=red!80!black, right= 30mm] {Spring (Vernal) Equinox};
        \node at (autumnEquinox) [tdplot_rotated_coords, circle, color=red!80!black, fill, inner sep=1.5pt] {};
        \node at (autumnEquinox) [tdplot_rotated_coords, color=red!80!black, left = 30mm] {Autumn Equinox};

% CIRCLE OF HORIZON, OBSERVER & ZENITH (BLUE)

        % DEFINE SECOND 3D COORDINATE FRAME / ORIENTATION (using tikz-3dplot)
        % This defines Srarxy, Starxz and Staryz that you may use.
        % Syntax: \tdplotsetcoord{Coordinate name without parentheses}{r}{\theta}{\phi}
        \tdplotsetcoord{Zenith}{\vectorRadiusZenith}{\vectorThetaZenith}{\vectorPhiZenith}

        % SET ZENITH COORDINATES
        % Set the rotated coordinate definition within display using a translation
        % Coordinate and Euler angles in the "z(\alpha)y(\beta)z(\gamma)" euler rotation convention
        % Syntax: \tdplotsetrotatedcoords{\alpha}{\beta}{\gamma}
        \tdplotsetrotatedcoords{\vectorPhiZenith}{\vectorThetaZenith}{0}

        % DRAW CIRCULAR PLANE HORIZON & LABEL (BLUE)
        \draw[tdplot_rotated_coords, fill, color=blue, opacity=.2] (\sphereRadius,0,\earthRadius) arc (0:360:\sphereRadius);
        \draw[tdplot_rotated_coords, dashed, thin, color=blue!80!black, opacity=1] (\sphereRadius,0,\earthRadius) arc (0:360:\sphereRadius);
        % Note that I use zenithRightAscension to figure out the think line :)
        \tdplotdrawarc[tdplot_rotated_coords, very thick, color=blue!80!black]{(0,0,\earthRadius)}{\sphereRadius}{-180+\phi-\zenithRightAscension}{\phi-\zenithRightAscension}{};
        \node at (0.8,-.9,.1) [tdplot_rotated_coords, color=blue!80!white] {Circle of Horizon};

        % DRAW LINE FROM OBSERVER TO ZENITH (BLUE)
        \draw[tdplot_rotated_coords, thick, color=blue, ->] (0,0,\earthRadius) -- (0,0,\sphereRadius);
        % \draw[tdplot_rotated_coords, thick, color=red] (Origin) -- (Zenith);

        % LABEL ZENITH (WITH BLUE DOT)
        \node at (0,0,\sphereRadius) [tdplot_rotated_coords, circle, color=blue, fill, inner sep=1.5pt] {};
        \node at (0,0,\sphereRadius) [tdplot_rotated_coords, color=blue!80!black, above = 5mm] {Zenith};

        % LABEL OBSERVER (WITH BLUE DOT)
        \node at (0,0,\earthRadius) [tdplot_rotated_coords, circle, color=blue!80!black, fill, inner sep=1.5pt] {};
        \node at (0,0,\earthRadius) [tdplot_rotated_coords, color=blue!80!black, right = 1mm] {Observer};

        % DRAW DASHED LINES ON HORIZON (Label N,S,E,W) (BLUE)
        \draw[tdplot_rotated_coords, dashed, thin, color=blue!80!white] (0,0,\earthRadius) -- (\sphereRadius,0,\earthRadius) node[color=black,anchor=south] {$S$};
        \draw[tdplot_rotated_coords, dashed, thin, color=blue!80!white] (0,0,\earthRadius) -- (-\sphereRadius,0,\earthRadius) node[color=black,anchor=north] {$N$};
        \draw[tdplot_rotated_coords, dashed, thin, color=blue!80!white] (0,0,\earthRadius) -- (0,\sphereRadius,\earthRadius) node[color=black,anchor=north] {$E$};
        \draw[tdplot_rotated_coords, dashed, thin, color=blue!80!white] (0,0,\earthRadius) -- (0,-\sphereRadius,\earthRadius) node[color=black,anchor=south] {$W$};

    % AZIMUTH & ALTITUDE (ORANGE)

        % *********************************************************
        % THESE CALCULATIONS ARE APPROX - I CAN'T FIGURE IT OUT
        % *********************************************************

        % DRAW THETA & PHI ARCS FOR STAR VECTOR (ORANGE)
        % Syntax: \tdplotdrawarc[coordinate frame, draw options]{center point}{r}{angle}{label options}{label}
        \tdplotdrawarc[tdplot_rotated_coords, very thick, color=orange, ->]{(0,0,\earthRadius)}{\sphereRadius}{-180}{-\zenithRightAscension+\starRightAscension}{very thick, right = 3mm, color=orange}{$Azimuth\qquad\qquad$}
        % Set the rotated coordinate system so the x'-y' plane lies within the
        % "theta plane" of the main coordinate system
        \tdplotsetthetaplanecoords{\vectorPhiStar}

        % I need to fudge the earth radius.  :(
        \tdplotdrawarc[tdplot_rotated_coords, very thick, color=orange, ->]{(0,0,0)}{\sphereRadius-.02}{180-\zenithDeclination-31}{\vectorThetaStar}{very thick, left = 1mm, color=orange}{$Altitude$}

    % STAR (WHITE)

        % DEFINE SECOND 3D COORDINATE FRAME / ORIENTATION (using tikz-3dplot)
        % This defines Srarxy, Starxz and Staryz that you may use.
        % Syntax: \tdplotsetcoord{Coordinate name without parentheses}{r}{\theta}{\phi}
        \tdplotsetcoord{Star}{\vectorRadiusStar}{\vectorThetaStar}{\vectorPhiStar}

        % SET STAR COORDINATES
        % Set the rotated coordinate definition within display using a translation
        % Coordinate and Euler angles in the "z(\alpha)y(\beta)z(\gamma)" euler rotation convention
        % Syntax: \tdplotsetrotatedcoords{\alpha}{\beta}{\gamma}
        \tdplotsetrotatedcoords{\vectorPhiStar}{\vectorThetaStar}{0}

        % DRAW VECTOR FROM ORIGIN TO STAR (WHITE)
        \draw[tdplot_rotated_coords, thick, color=white, ->] (Origin) -- (0,0,\sphereRadius);
        % \draw[thick, color=white] (Origin) -- (Star);
        % Add some dashed line to make it easier to see
        \draw[tdplot_rotated_coords, dashed, thick, color=white] (Starxy) -- (Star);
        \draw[tdplot_rotated_coords, dashed, thick, color=white] (Origin) -- (Starxy);

        % DRAW & LABEL STAR (usetikzlibrary{positioning}) (WHITE)
        \node at (Star) [tdplot_screen_coords, circle, color=white, fill, inner sep=1.5pt] {};
        \node at (Star) [tdplot_screen_coords, color=white, above = 1mm] {$Star$};

    % RIGHT ASCENSION & DECLINATION (WHITE)
    
        % DRAW THETA & PHI ARCS FOR STAR VECTOR (WHITE)
        % Syntax: \tdplotdrawarc[coordinate frame, draw options]{center point}{r}{angle}{label options}{label}
        \tdplotdrawarc[very thick, color=white, ->]{(Origin)}{\sphereRadius}{0}{\vectorPhiStar}{very thick, above = 3mm, color=white}{$Right Ascension\qquad\qquad$}
        % Set the rotated coordinate system so the x'-y' plane lies within the
        % "theta plane" of the main coordinate system
        % Syntax: \tdplotsetthetaplanecoords{\phi}
        \tdplotsetthetaplanecoords{\vectorPhiStar}
        \tdplotdrawarc[tdplot_rotated_coords, very thick, color=white, ->]{(Origin)}{\sphereRadius}{90}{\vectorThetaStar}{very thick, left = 1mm, color=white}{$Declination$}

    % MERIDIAN (ORANGE)

        % SET MERIDIAN COORDINATES
        % Set the rotated coordinate definition within display using a translation
        % Coordinate and Euler angles in the "z(\alpha)y(\beta)z(\gamma)" euler rotation convention
        % Syntax: \tdplotsetrotatedcoords{\alpha}{\beta}{\gamma}
        %%%\tdplotsetrotatedcoords{\vectorPhiMeridian}{\vectorThetaMeridian}{0}

        % DRAW  MERIDIAN & LABEL (ORANGE)
        %%%\draw[tdplot_rotated_coords, dashed, thick, color=orange!80!black] (\sphereRadius,0,0) arc (0:360:\sphereRadius);
        %%%\tdplotdrawarc[tdplot_rotated_coords, very thick, color=orange!80!black]{(0,0,0)}{\sphereRadius}{30}{215}{};
        %%%\node at (.3,-.4,-3.3) [tdplot_rotated_coords, color=orange!80!black] {Meridian};

    % DRAW CIRCLE AROUND SPHERE (BLACK)
        \draw[tdplot_screen_coords, line width=1mm, color=black] (0,0) circle (\sphereRadius);

\end{tikzpicture}

\end{document}