\documentclass{standalone}
\usepackage{tikz}
\usepackage[siunitx, RPvoltages]{circuitikz}

\usetikzlibrary{positioning, arrows, shapes.gates.logic.US, shapes.gates.logic.IEC, calc}

% This is the d flip=flop from the circuitikz package
\tikzset{
        my-d-flip-flop/.style={
            flipflop D, external pins width=0, thick
        },
}

\begin{document}

\resizebox{7cm}{5cm}{

    \begin{circuitikz}[scale=1.0, transform shape]

        % DRAW NODES (I like to do them relative to each other)
        \node [my-d-flip-flop]  (DFF)  at (0,0)                    {\normalsize $dff_0$};
        \node []                (D)    at ($(DFF)+(-2.0,.85)$)     {\normalsize $d$};
        \node []                (CLK)  at ($(DFF)+(-2.2,-.85)$)    {\normalsize $clk$};
        \node []                (Q)    at ($(DFF)+(2.0,.85)$)      {\normalsize $q$};
        \node []                (QBAR) at ($(DFF)+(2.4,-.85)$)     {\normalsize $q\_bar$};
        \node []                (NAME) at ($(DFF)+(0, 2.0)$)       {\Large \textbf {D FLIP-FLOP}};

        % CONNECT INPUTS
        \draw [semithick] (D) -- (DFF.pin 1);
        \draw [semithick] (CLK) -- (DFF.pin 3);

        % CONNECT OUTPUTS
        \draw [semithick] (DFF.pin 6) -- (Q);
        \draw [semithick] (DFF.pin 4) -- (QBAR);

     \end{circuitikz}

}

\end{document} 
