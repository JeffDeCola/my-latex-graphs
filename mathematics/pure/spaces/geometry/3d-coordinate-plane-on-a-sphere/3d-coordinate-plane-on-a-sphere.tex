\documentclass[border=5mm]{standalone}
\usepackage{tikz}
\usepackage{tikz-3dplot}
\usetikzlibrary{positioning}

\begin{document}

% DEFINE YOUR 3D COORDINATE FRAME / ORIENTATION (using tikz-3dplot)
% Will set the orientation of x,y,z
% Set polar coordinate. theta ( z -> phi) phi (x -> y) 
% Syntax: \tdplotsetdisplay{\theta_d}{\phi_d}
\tdplotsetmaincoords{70}{120}

% SET SOME VARIABLES

    % Sphere radius
    \pgfmathsetmacro{\sphereRadius}{1}

    % Set (2D) Polar Coordinates for New Coordinate System
    % For red vector
    \pgfmathsetmacro{\vectorRadiusP}{1}
    \pgfmathsetmacro{\vectorThetaP}{25}
    \pgfmathsetmacro{\vectorPhiP}{45}
    % For green vector
    \pgfmathsetmacro{\vectorRadiusQ}{1}
    \pgfmathsetmacro{\vectorThetaQ}{45}
    \pgfmathsetmacro{\vectorPhiQ}{80}

% USE tdplot_main_coords FOR YOUR DEFINED 3D COORDINATE FRAME
% Coordinate transformation provided by tikz-3dplot
\begin{tikzpicture}[scale=6, tdplot_main_coords]

    % DEFINE coordinates - the origin
    \coordinate (Origin) at (0,0,0);

    % DRAW SPHERE ON 3D COORDINATE SYSTEM
    \shadedraw[tdplot_screen_coords] (0,0) circle (\sphereRadius);

    % DRAW SPHERE EQUATOR
    \draw[thin, fill, color=white] (\sphereRadius,0,0) arc (0:360:\sphereRadius);
    \draw[thin, color=black] (\sphereRadius,0,0) arc (0:360:\sphereRadius);

    % DRAW MAIN COORDINATE SYSTEM AXES - X, Y, Z AXIS WITH LABEL (NODE)
    % Syntax: \draw[characteristics] (start point) -- (end point) node [anchor=position] {label};
    \draw[thick, color=black, ->] (Origin) -- (\sphereRadius+.5,0,0) node[anchor=north east] {$x$};
    \draw[thick, color=black, ->] (Origin) -- (0,\sphereRadius+.5,0) node[anchor=north west] {$y$};
    \draw[thick, color=black, ->] (Origin) -- (0,0,\sphereRadius+.5) node[anchor=south] {$z$};
    \draw[dashed, color=black, ->] (Origin) -- (-\sphereRadius-.5,0,0);
    \draw[dashed, color=black, ->] (Origin) -- (0,-\sphereRadius-.5,0);
    \draw[dashed, color=black, ->] (Origin) -- (0,0,-\sphereRadius-.5);

    % DRAW DOTS ON COORDINATES INTERSECTING SPHERE
    \node at (\sphereRadius,0,0) [circle, color=black, fill, inner sep=1.5pt] {};
    \node at (0,\sphereRadius,0) [circle, color=black, fill, inner sep=1.5pt] {};
    \node at (0,0,\sphereRadius) [circle, color=black, fill, inner sep=1.5pt] {};
    \node at (-\sphereRadius,0,0) [circle, color=black, fill, inner sep=1.5pt] {};
    \node at (0,-\sphereRadius,0) [circle, color=black, fill, inner sep=1.5pt] {};
    \node at (0,0,-\sphereRadius) [circle, color=black, fill, inner sep=1.5pt] {};

    % DEFINE SECOND 3D COORDINATE FRAME / ORIENTATION (using tikz-3dplot)
    % This defines Pxy, Pxz and Pyz that you may use.
    % Syntax: \tdplotsetcoord{Coordinate name without parentheses}{r}{\theta}{\phi}
    \tdplotsetcoord{P}{\vectorRadiusP}{\vectorThetaP}{\vectorPhiP}
    \tdplotsetcoord{Q}{\vectorRadiusQ}{\vectorThetaQ}{\vectorPhiQ}

    % DRAW READ AND GREEN VECTORS FROM ORIGIN TO POINT P & Q
    \draw[-stealth, very thick, color=red!80!black] (Origin) -- (P);
    \draw[-stealth, very thick, color=green!60!black] (Origin) -- (Q);
    % Draw dashed line
    \draw[dashed, color=red!80!black] (Origin) -- (Pxy);
    \draw[dashed, color=red!80!black] (Pxy) -- (P);
    \draw[dashed, color=green!60!black] (Origin) -- (Qxy);
    \draw[dashed, color=green!60!black] (Qxy) -- (Q);

    % LABEL POINT P AND Q (usetikzlibrary{positioning})
    \node at (P) [tdplot_screen_coords, circle, color=red!80!black, fill, inner sep=1.5pt] {};
    \node at (P) [tdplot_screen_coords, color=red!80!black, left = 1mm] {$P$};
    \node at (Q) [tdplot_screen_coords, circle, color=green!60!black, fill, inner sep=1.5pt] {};
    \node at (Q) [tdplot_screen_coords, color=green!60!black, right = 1mm] {$Q$};

    % DRAW THETA & PHI ARCS FOR RED VECTOR P
    % Syntax: \tdplotdrawarc[coordinate frame, draw options]{center point}{r}{angle}{label options}{label}
    \tdplotdrawarc[color=red!80!black]{(Origin)}{\sphereRadius/5}{0}{\vectorPhiP}{anchor=north, color=red!80!black}{$\phi_A$}
    % Set the rotated coordinate system so the x'-y' plane lies within the
    % "theta plane" of the main coordinate system
    % Syntax: \tdplotsetthetaplanecoords{\phi}
    \tdplotsetthetaplanecoords{\vectorPhiP}
    \tdplotdrawarc[tdplot_rotated_coords, color=red!80!black]{(Origin)}{\sphereRadius/5}{90}{\vectorThetaP}{anchor=south west, color=red!80!black}{$\theta_A$}

    % DRAW THETA & PHI ARCS FOR GREEN VECTOR Q
    % Syntax: \tdplotdrawarc[coordinate frame, draw options]{center point}{r}{angle}{label options}{label}
    \tdplotdrawarc[color=green!60!black]{(Origin)}{\sphereRadius/2}{0}{\vectorPhiQ}{anchor=north, color=green!60!black}{$\phi_A$}
    % Set the rotated coordinate system so the x'-y' plane lies within the
    % "theta plane" of the main coordinate system
    % Syntax: \tdplotsetthetaplanecoords{\phi}
    \tdplotsetthetaplanecoords{\vectorPhiQ}
    \tdplotdrawarc[tdplot_rotated_coords, color=green!60!black]{(Origin)}{\sphereRadius/2}{90}{\vectorThetaQ}{anchor=south west, color=green!60!black}{$\theta_A$}

    % DRAW RED AND GREEN CIRCLES (ARCS) ON SPHERE FOR VECTORS
    \tdplotsetthetaplanecoords{\vectorPhiP}
    \draw[tdplot_rotated_coords, dashed, color=red] (\sphereRadius,0,0) arc (0:231:\sphereRadius);
    \draw[tdplot_rotated_coords, dashed, color=red] (\sphereRadius,0,0) arc (360:270:\sphereRadius);
    \draw[tdplot_rotated_coords, thick, color=red] (\sphereRadius,0,0) arc (0:151:\sphereRadius);
    \draw[tdplot_rotated_coords, very thick, color=blue] (\sphereRadius,0,0) arc (0:\vectorThetaP:\sphereRadius);
    \draw[tdplot_rotated_coords, thick, color=red] (\sphereRadius,0,0) arc (360:336:\sphereRadius);
    \tdplotsetthetaplanecoords{\vectorPhiQ}
    \draw[tdplot_rotated_coords, dashed, color=green] (\sphereRadius,0,0) arc (0:241:\sphereRadius);
    \draw[tdplot_rotated_coords, dashed, color=green] (\sphereRadius,0,0) arc (360:270:\sphereRadius);
    \draw[tdplot_rotated_coords, thick, color=green] (\sphereRadius,0,0) arc (0:140:\sphereRadius);
    \draw[tdplot_rotated_coords, very thick, color=blue] (\sphereRadius,0,0) arc (0:\vectorThetaQ:\sphereRadius);
    \draw[tdplot_rotated_coords, thick, color=green] (\sphereRadius,0,0) arc (360:330:\sphereRadius);

    % DRAW A BLUE ARC FROM P TO Q ON SPHERE
    % 0,0,0 would be on the equator.
    % Syntax: The angle in degrees rotated around x,y,z axis.
    % I HAVE NO IDEA HOW TO CALCULATE THIS - THIS WOULD BE A GREAT EXERCISE
    % I JUST PLUGGED IN NUMBERS TO APPROX IT
    % BUT I WOULD LIKE TO KNOW HOW TO DO THIS
    \tdplotsetrotatedcoords{13.5}{-68.5}{12.5}
    \draw[tdplot_rotated_coords, thick, color=blue] (\sphereRadius,0,0) arc (0:27:\sphereRadius);

\end{tikzpicture}

\end{document}