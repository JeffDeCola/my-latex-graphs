\documentclass[border=5mm]{standalone}
\usepackage{tikz}
\usepackage{tikz-3dplot}
\usetikzlibrary{positioning}

\begin{document}

% DEFINE YOUR 3D COORDINATE FRAME / ORIENTATION (using tikz-3dplot)
% Will set the orientation of x,y,z
% Set polar coordinate. theta ( z -> phi) phi (x -> y) 
% Syntax: \tdplotsetdisplay{\theta_d}{\phi_d}
\tdplotsetmaincoords{70}{120}

% SET SOME VARIABLES

    % Set sphere radius
    \pgfmathsetmacro{\sphereRadius}{1}
    
    % Set Polar (2D) Coordinates for New Coordinate System
    \pgfmathsetmacro{\vectorRadiusP}{.8}
    \pgfmathsetmacro{\vectorThetaP}{30}
    \pgfmathsetmacro{\vectorPhiP}{70}

% USE tdplot_main_coords FOR YOUR DEFINED 3D COORDINATE FRAME
% Coordinate transformation provided by tikz-3dplot
\begin{tikzpicture}[scale=6, tdplot_main_coords]

    % *************************************************************
    % PART I - MAIN COORDINATE SYSTEM
    % *************************************************************

    % DEFINE coordinates - the origin
    \coordinate (Origin) at (0,0,0);

    % DRAW SPHERE ON 3D COORDINATE SYSTEM
    \shadedraw[tdplot_screen_coords, ball color = white] (0,0) circle (\sphereRadius);

    % DRAW SPHERE EQUATOR
    \draw[thin, fill, color=white] (\sphereRadius,0,0) arc (0:360:\sphereRadius);
    \draw[thin, color=black] (\sphereRadius,0,0) arc (0:360:\sphereRadius);
    
    % DRAW MAIN COORDINATE SYSTEM AXES - X, Y, Z AXIS WITH LABEL (NODE)
    % Syntax: \draw[characteristics] (start point) -- (end point) node [anchor=position] {label};
    \draw[thick, color=black, ->] (Origin) -- (\sphereRadius+.5,0,0) node[anchor=north east] {$x$};
    \draw[thick, color=black, ->] (Origin) -- (0,\sphereRadius+.5,0) node[anchor=north west] {$y$};
    \draw[thick, color=black, ->] (Origin) -- (0,0,\sphereRadius+.5) node[anchor=south] {$z$};
    
    % DRAW DOTS ON COORDINATES INTERSECTING SPHERE
    \node at (\sphereRadius,0,0) [circle, color=black, fill, inner sep=1.5pt] {};
    \node at (0,\sphereRadius,0) [circle, color=black, fill, inner sep=1.5pt] {};
    \node at (0,0,\sphereRadius) [circle, color=black, fill, inner sep=1.5pt] {};

    % DEFINE SECOND 3D COORDINATE FRAME / ORIENTATION (using tikz-3dplot)
    % This defines Pxy, Pxz and Pyz that you may use.
    % Syntax: \tdplotsetcoord{Coordinate name without parentheses}{r}{\theta}{\phi}
    \tdplotsetcoord{P}{\vectorRadiusP}{\vectorThetaP}{\vectorPhiP}

    % LABEL POINT P (usetikzlibrary{positioning})
    \node at (P) [tdplot_screen_coords, circle, color=black, fill, inner sep=1.5pt] {};
    \node at (P) [tdplot_screen_coords, color=black, above left = 1mm] {$P$};

    % DRAW RED VECTOR FROM ORIGIN TO POINT P
    \draw[very thick, color=red, ->] (Origin) -- (P);
    % Draw dashed line
    \draw[dashed, color=red] (Origin) -- (Pxy);
    \draw[dashed, color=red] (Pxy) -- (P);

    % DRAW THETA & PHI ANGLES TO RED VECTOR
    % Syntax: \tdplotdrawarc[coordinate frame, draw options]{center point}{r}{angle}{label options}{label}
    \tdplotdrawarc[color=red]{(Origin)}{\sphereRadius/5}{0}{\vectorPhiP}{anchor=north, color=red}{$\phi$}
    % Set the rotated coordinate system so the x'-y' plane lies within the
    % "theta plane" of the main coordinate system
    % Syntax: \tdplotsetthetaplanecoords{\phi}
    \tdplotsetthetaplanecoords{\vectorPhiP}
    \tdplotdrawarc[tdplot_rotated_coords, color=red]{(Origin)}{\sphereRadius/4}{0}{\vectorThetaP}{anchor=south west, color=red}{$\theta$}

    % *************************************************************
    % PART II - SECOND COORDINATE SYSTEM
    % *************************************************************

    % Set the rotated coordinate definition within display using a translation
    % Coordinate and Euler angles in the "z(\alpha)y(\beta)z(\gamma)" euler rotation convention
    % Syntax: \tdplotsetrotatedcoords{\alpha}{\beta}{\gamma}
    \tdplotsetrotatedcoords{\vectorPhiP}{\vectorThetaP}{0}

    % TRANSLATE ROTATED COORDINATE SYSTEM TO THE POINT P
    % Syntax: \tdplotsetrotatedcoordsorigin{point}
    \tdplotsetrotatedcoordsorigin{(P)}
    
    % DRAW SECOND WHITE EQUATOR CIRCLE
    \draw[tdplot_rotated_coords, thin, fill, color=white, opacity=0.5] (\sphereRadius/3,0,0) arc (0:360:\sphereRadius/3);
    \draw[tdplot_rotated_coords, thin, color=black] (\sphereRadius/3,0,0) arc (0:360:\sphereRadius/3);
    
    % DRAW SECOND COORDINATE SYSTEM AXES (BLUE) - X, Y, Z AXIS
    \draw[tdplot_rotated_coords, thick, color=blue, , ->] (0,0,0) -- (\vectorRadiusP/1.6,0,0) node[anchor=north west]{$x'$};
    \draw[tdplot_rotated_coords, thick, color=blue, ->] (0,0,0) -- (0,\vectorRadiusP/1.6,0) node[anchor=west]{$y'$};
    \draw[tdplot_rotated_coords, thick, color=blue, ->] (0,0,0) -- (0,0,\vectorRadiusP/1.6) node[anchor=south]{$z'$};

    % DRAW GREEN VECTOR FROM ORIGIN TO POINT P (CAN'T DO WHAT WE DID ABOVE)
    \draw[tdplot_rotated_coords, very thick,color=green, ->] (0,0,0) -- (.2,.2,.2);
    % Draw dashed line
    \draw[tdplot_rotated_coords, dashed,color=green] (0,0,0) -- (.2,.2,0);
    \draw[tdplot_rotated_coords, dashed,color=green] (.2,.2,0) -- (.2,.2,.2);

    % DRAW THETA & PHI ANGLES TO ORANGE VECTOR
    % Syntax: \tdplotdrawarc[coordinate frame, draw options]{center point}{r}{angle}{label options}{label}
    \tdplotdrawarc[tdplot_rotated_coords, color=green]{(0,0,0)}{\sphereRadius/5}{0}{45}{anchor=north, color=green}{$\phi$}
    % Set the rotated coordinate system so the x'-y' plane lies within the
    % "theta plane" of the main coordinate system
    % Syntax: \tdplotsetthetaplanecoords{\phi}
    \tdplotsetrotatedthetaplanecoords{45}
    \tdplotdrawarc[tdplot_rotated_coords, color=green]{(0,0,0)}{\sphereRadius/5}{0}{55}{anchor=south west, color=green}{$\theta$}

\end{tikzpicture}

\end{document}